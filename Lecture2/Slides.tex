\documentclass[10pt,pdf,utf8,russian,aspectratio=169]{beamer}
\usepackage[T2A]{fontenc}
\usetheme{Copenhagen}
\usepackage{setspace}
\usepackage{amsmath}
\usepackage{pgfplots}
\usepackage[utf8]{inputenc}
\usepackage{tikz-cd}
\usepackage[all, 2cell]{xy}
\usepackage{amssymb}
\usepackage{verba tim}
\usepackage[all]{xy}
\usepackage{tikz}
\usepackage{bussproofs}
\usepackage{dsfont}
\usepackage{mathabx}
\usepackage{animate}
\usetikzlibrary{graphs}
\usetikzlibrary{arrows}
\usepackage{hyperref}
\usepackage[english,russian]{babel}
\usepackage{listings}
\usepackage{color}
\usepackage{tikz}
\usepackage{listings}
\pgfplotsset{compat=1.16}
\newtheorem{defin}{Definition}
\newtheorem{theor}{Theorem}
\newtheorem{prop}{Proposition}
\title{Functional programming, Seminar No. 2}
\author{Danya Rogozin \\ Lomonosov Moscow State University, \\ Serokell O\"{U}}
\date{Higher School of Economics \\ Faculty of Computer Science}
\begin{document}
\lstset{
  frame=none,
  xleftmargin=2pt,
  stepnumber=1,
  numbers=left,
  numbersep=5pt,
  numberstyle=\ttfamily\tiny\color[gray]{0.3},
  belowcaptionskip=\bigskipamount,
  captionpos=b,
  escapeinside={*'}{'*},
  language=haskell,
  tabsize=2,
  emphstyle={\bf},
  commentstyle=\it,
  stringstyle=\mdseries\rmfamily,
  showspaces=false,
  keywordstyle=\bfseries\rmfamily,
  columns=flexible,
  basicstyle=\small\sffamily,
  showstringspaces=false,
  morecomment=[l]\%,
}
\maketitle

\begin{frame}[fragile]
  \frametitle{Bindings}

  The equality sign in Haskall denotes binding:

    \begin{lstlisting}[language=Haskell]
      fortyTwo = 42
      coolString = "coolString"
    \end{lstlisting}

\vspace{\baselineskip}

Local binding with the \verb"let"-keyword:
  \begin{lstlisting}[language=Haskell]
    fortyTwo = let number = 43 in number - 1
  \end{lstlisting}
\end{frame}

\begin{frame}[fragile]
  \frametitle{Function definitions}

  Functions are also defined as bindings:

  \begin{lstlisting}[language=Haskell]
    add x y = x + y
    userName name = "Username: " ++ name
    id x = x
  \end{lstlisting}

  \vspace{\baselineskip}

  The same functions defined via lambda:
  \begin{lstlisting}[language=Haskell]
    add = \x y -> x + y
    userName = \name -> "Username: " ++ name
    id = \x -> x
  \end{lstlisting}
\end{frame}

\begin{frame}[fragile]
  \frametitle{Function application}

  As in lambda calculus, function application is right associative by default

  \begin{lstlisting}[language=Haskell]
    {-
    foo x y z = f x y z = ((f x) y) z
    -}
  \end{lstlisting}

\vspace{\baselineskip}

  One may use the dollar infix operator in order to avoid brackets overuse. For example, the following functions have exactly the same behaviour:

  \begin{lstlisting}[language=Haskell]
    function f x y z = f ((x y) z)
    function1 f x y z = f $ x y $ z
  \end{lstlisting}
\end{frame}

\begin{frame}[fragile]
  \frametitle{Immutability and laziness}

In Haskell, values are immutable. For example, the following code doesn't terminate since it yields an infinite loop in contrast to Python or JS:

  \begin{lstlisting}[language=Haskell]
    x = 5
    x = x + x
  \end{lstlisting}

\vspace{\baselineskip}

The following program doesn't fail since our semantics is lazy:

  \begin{lstlisting}[language=Haskell]
    seventyTwo = const 72 undefined
  \end{lstlisting}
\end{frame}

\begin{frame}[fragile]
  \frametitle{Recursion}

  The straightforward recursion:
  \begin{lstlisting}[language=Haskell]
    factorial n = if n == 0 then 1 else n * factorial (n - 1)
  \end{lstlisting}

  \vspace{\baselineskip}

  The factorial function implemented via so-called tail recursion:
  \begin{lstlisting}[language=Haskell]
    tailFactorial n = helper 1 n
      where
      helper acc x =
        if x > 1
        then then helper (acc * x) (x - 1)
        else acc
  \end{lstlisting}
\end{frame}

\begin{frame}[fragile]
  \frametitle{Guards}

  Let us take a look at the factorial implementation via guards:

  \begin{lstlisting}[language=Haskell]
    tailFactorial n = helper 1 n
      where
      helper acc x | x > 1 = helper (acc * x) (x - 1)
                   | otherwise acc
  \end{lstlisting}
\end{frame}

\begin{frame}
  \frametitle{Currying and partial application}
\end{frame}

\begin{frame}[fragile]
  \frametitle{Basic datatypes}
  The basic datatypes are:
  \begin{itemize}
    \item \verb"Bool": Boolean values
    \item \verb"Int": Bounded integer datatype
    \item \verb"Integer": Unbounded integer datatype
    \item \verb"Char": Unicode characters
    \item \verb"()": Unit value datatype
    \item If \verb"a" and \verb"b" are types, then \verb"a -> b" is a type
    \item If \verb"a" and \verb"b" are types, then \verb"(a,b)" is a type
    \item If \verb"a" is a type, then \verb"[a]" is a type
  \end{itemize}

  A type declaration has the following form:

  \begin{lstlisting}[language=Haskell]
    term :: type
  \end{lstlisting}
\end{frame}

\begin{frame}[fragile]
  \frametitle{Function declaration with datatypes}

  Let us recall the examples of function declarations:

    \begin{lstlisting}[language=Haskell]
      add x y = x + y
      userName name = "Username: " ++ name
      id x = x
    \end{lstlisting}

    \vspace{\baselineskip}

    One may annotate these functions with types as follows:

    \begin{lstlisting}[language=Haskell]
      add :: Int -> Int -> Int
      add x y = x + y

      userName :: String -> String
      userName name = "Username: " ++ name

      id :: Char -> Char
      id x = x
    \end{lstlisting}

\vspace{\baselineskip}

  Note that such calls as \verb"userName 5" or \verb"id ''hello stewart''" cause type errors.

\end{frame}

\end{document}

\documentclass[10pt,pdf,utf8,russian,aspectratio=169]{beamer}
\usepackage[T2A]{fontenc}
\usetheme{Copenhagen}
\usepackage{setspace}
\usepackage{amsmath}
\usepackage{pgfplots}
\usepackage[utf8]{inputenc}
\usepackage{tikz-cd}
\usepackage[all, 2cell]{xy}
\usepackage{amssymb}
\usepackage{verba tim}
\usepackage[all]{xy}
\usepackage{tikz}
\usepackage{bussproofs}
\usepackage{dsfont}
\usepackage{mathabx}
\usepackage{animate}
\usetikzlibrary{graphs}
\usetikzlibrary{arrows}
\usepackage{hyperref}
\usepackage[english,russian]{babel}
\usepackage{listings}
\usepackage{color}
\usepackage{tikz}
\usepackage{listings}
\newtheorem{defin}{Definition}
\newtheorem{theor}{Theorem}
\newtheorem{prop}{Proposition}
\title{Functional programming, Seminar No 1}
\author{Danya Rogozin \\ Lomonosov Moscow State University, \\ Serokell O\"{U}}
\date{Higher School of Economics \\ Faculty of Computer Science}
\begin{document}
\maketitle

\begin{frame}
  \frametitle{General words on Haskell}
\end{frame}

\begin{frame}
  \frametitle{GHC}
\end{frame}

\begin{frame}
  \frametitle{GHCi}
\end{frame}

\begin{frame}
  \frametitle{Cabal}
\end{frame}

\begin{frame}
  \frametitle{Stack}
\end{frame}

\begin{frame}
  \frametitle{Snapshots}
\end{frame}

\begin{frame}
  \frametitle{Creating a Haskell project via Stack}
\end{frame}

\begin{frame}
  \frametitle{\verb"stack.yaml" and cabal files}
\end{frame}

\begin{frame}
  \frametitle{Building and running a project}
\end{frame}

\begin{frame}
  \frametitle{Hackage}
\end{frame}

\begin{frame}
  \frametitle{Hoogle}
\end{frame}

\begin{frame}
  \frametitle{Summary}
\end{frame}

\end{document}

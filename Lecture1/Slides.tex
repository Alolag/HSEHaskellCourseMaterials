\documentclass[10pt,pdf,utf8,russian,aspectratio=169]{beamer}
\usepackage[T2A]{fontenc}
\usetheme{Copenhagen}
\usepackage{setspace}
\usepackage{amsmath}
\usepackage{pgfplots}
\usepackage[utf8]{inputenc}
\usepackage{tikz-cd}
\usepackage[all, 2cell]{xy}
\usepackage{amssymb}
\usepackage{verba tim}
\usepackage[all]{xy}
\usepackage{tikz}
\usepackage{bussproofs}
\usepackage{dsfont}
\usepackage{mathabx}
\usepackage{animate}
\usetikzlibrary{graphs}
\usetikzlibrary{arrows}
\usepackage{hyperref}
\usepackage[english,russian]{babel}
\usepackage{listings}
\usepackage{color}
\usepackage{tikz}
\usepackage{listings}
\newtheorem{defin}{Definition}
\newtheorem{theor}{Theorem}
\newtheorem{prop}{Proposition}
\title{Functional programming, Seminar No. 1}
\author{Danya Rogozin \\ Lomonosov Moscow State University, \\ Serokell O\"{U}}
\date{Higher School of Economics \\ Faculty of Computer Science}
\begin{document}
\maketitle

\begin{frame}
  \frametitle{General words on Haskell}

  \begin{itemize}
    \item The language is named after an American logician Haskell Curry
    \item First implementation: 1990
    \item The language standard: Haskell2010
    \item Default compiler: Glasgow Haskell compiler
    \item Haskell is a strongly-typed, polymorphic, and purely functional programming language
    \onslide<2->{
    \item This course is quite introductory. We partially piggyback the ITMO Haskell course.
    }
    \onslide<3->{
    \item Vox populi:
    \begin{center}
    \includegraphics[scale=0.5]{Pics/VoxPopuli.png}
    \end{center}
    }
  \end{itemize}
\end{frame}

\begin{frame}
  \frametitle{The Haskell Platform installation}

  There are several ways to install the Haskell platform on Mac:

\onslide<1->{
  \begin{itemize}
    \item Download the \verb".pkg" file and install the corresponding package}
    \onslide<2->{\item Run the script \verb"curl -sSL https://get.haskellstack.org/ | sh"}
    \onslide<3->{\item Install ghc, stack, and cabal via Homebrew
  \end{itemize}
  }

\onslide<4->{
  Choose any way you prefer. All these ways are equivalent to each other.
}
  \vspace{\baselineskip}


\onslide<5->{
 I'm a Mac user, but I believe that you'll manage to install the Haskell Platform on NixOs/Windows/Linux/etc quite quickly.
 }
\end{frame}

\begin{frame}
  \frametitle{GHC}

\onslide<1->{
  \begin{itemize}
    \item GHC is a default Haskell compiler as we told above
    \item GHC is an open-source project. Don't hesistate to contribute!
    \item GHC is mostly implemented on Haskell
    }
    \item GHC is developed under the GHC Steering committee control
    \onslide<2->{
    \item Very roughly, compiling pipeline is arranged as follows: \\ parsing $\Rightarrow$ compile-time (type-checking mostly)
    $\Rightarrow$ runtime (program execution)
    }
  \end{itemize}
\end{frame}

\begin{frame}
  \frametitle{GHCi}

  \begin{itemize}
    \item GHCi is a Haskell interpreter based on GHC
    \item One may run GHCi with a quite simple command \verb"ghci" on a shell
    \item You play with GHCi as a calculator, the ordinary arithmetical operators are written in a usual way
    \item Take a look at the GHCi chapter in the GHC User's Guide to be familiar with GHCi closely
  \end{itemize}

  \begin{center}
  \includegraphics[scale=0.5]{Pics/GHCi.png}
  \end{center}
\end{frame}

\begin{frame}
  \frametitle{Cabal}

\onslide<1->{
  \begin{itemize}
    \item Cabal is a system of library and dependency management
    \item A \verb".cabal" file describes the version of a package and its dependencies
    \item Cabal is also a packaging tool
    \item Keep in mind that Cabal is known as a reason of so-called dependency hell
  \end{itemize}}

\onslide<2->{
  That's how this dependency hell might look like:

  \begin{center}
  \includegraphics[scale=0.18]{Pics/Bosch.jpg}
  \end{center}
}
\end{frame}

\begin{frame}
  \frametitle{Stack}

\onslide<1->{
  \begin{itemize}
    \item Stack is a cross-platform build tool for Haskell projects
    \item Stack allows one to
    }
  \onslide<2->{
    \begin{itemize}
      \item install packages and version of GHC (and their concrete versions) you need
      }
  \onslide<3->{
      \item build, execute, and test projects
      }
  \onslide<4->{
      \item reproduce builds
      }
  \onslide<5->{
      \item create an isolated location
    \end{itemize}
  \end{itemize}
  }
\end{frame}

\begin{frame}
  \frametitle{Snapshots}

  \begin{itemize}
    \onslide<1->{
    \item Snapshot is a curated package set used by Stack
    }
    \onslide<2->{
    \item Stackage is a stable repository that stores snapshots}
    \onslide<3->{\item Resolver is a reference to a required snapshot
    }
    \onslide<4->{
    \item Let us take a look at the screenshot from Stackage:

    \begin{center}
    \includegraphics[scale=0.3]{Pics/Snapshots.png}
    \end{center}
    }
  \end{itemize}
\end{frame}

\begin{frame}
  \frametitle{Ecosystem encapsulation}

  The Haskell ecosystem encapsulation might be described as the following sequence:

  \begin{center}
  \includegraphics[scale=0.3]{Pics/Eco.png}
  \end{center}
\end{frame}

\begin{frame}
  \frametitle{Creating a Haskell project via Stack}
  \onslide<1->{
  \begin{itemize}
    \item Figure out how to call your project and run the script \verb"stack new <projectname>"
    \item You will see the following story after the command \verb"tree ." in the project directory:
    }
  \onslide<2->{

  \begin{center}
  \includegraphics[scale=0.35]{Pics/Tree.png}
  \end{center}
  \end{itemize}
  }
\end{frame}

\begin{frame}
  \frametitle{\verb"stack.yaml"}

\onslide<1->{
  Let us discuss dependencies files in a Haskell project. First of all, we observe the \verb"stack.yaml" file:
}

\onslide<2->{
\begin{center}
\includegraphics[scale=0.35]{Pics/StackYaml.png}
\end{center}
}
\end{frame}

\begin{frame}
  \frametitle{Cabal file}

\onslide<1->{
  As we told above, the \verb".cabal" file describe the relevant version of a project and its dependencies:
}
\onslide<2->{
\begin{center}
\includegraphics[scale=0.35]{Pics/CabalFile.png}
\end{center}
  }
\end{frame}

\begin{frame}
  \frametitle{\verb"package.yaml"}

\onslide<1->{
  The \verb"package.yaml" generates automatically from the \verb"stack.yaml" and \verb".cabal" files:
}

\onslide<2->{
\begin{center}
\includegraphics[scale=0.36]{Pics/PackageYaml.png}
\end{center}
  }
\end{frame}

\begin{frame}
  \frametitle{Building and running a project}

\onslide<1->{
The following commands are crucially important:
}

\onslide<2->{
  \begin{itemize}
    \item \verb"stack build"}
    \onslide<3->{\item \verb"stack run"}
    \onslide<4->{\item \verb"stack exec"}
    \onslide<5->{\item \verb"stack ghci"}
    \onslide<6->{\item \verb"stack clean"}
    \onslide<7->{\item \verb"stack test"
  \end{itemize}
  }
\onslide<8->{
  The roles of these commands follow their names which are quite self-explanatory.
  }
\end{frame}

\begin{frame}
  \frametitle{Hackage}

\onslide<1->{
  According to its description, 'Hackage is the Haskell community's central package archive of open source software`.
}

\onslide<2->{
  \begin{itemize}
    \item Webpage: \verb"https://hackage.haskell.org"
    }
\onslide<3->{
    \item Browsing packages, simplified package search, current uploads.
  \end{itemize}
  }
\onslide<4->{
\begin{center}
\includegraphics[scale=0.23]{Pics/HackageExample.png}
\end{center}
  }
\end{frame}

\begin{frame}
  \frametitle{Hoogle}

\onslide<1->{
  Hoogle is a sort of Haskell search engine. Webpage: \verb"https://hoogle.haskell.org".
}

\onslide<2->{
\begin{center}
\includegraphics[scale=0.3]{Pics/Hoogle.png}
\end{center}
}
\end{frame}

\begin{frame}
  \frametitle{Summary}

\onslide<1->{
  We observed today such topics as

  \begin{enumerate}
    \item General aspects of GHC and GHCi
    \item The Haskell Platform installation
    \item Dependency management via Stack and Cabal
  }
  \onslide<2->{
    \item In other words, the Haskell ecosystem in a nutshell
  \end{enumerate}
  }

  \vspace{\baselineskip}

\onslide<3->{
  On the next seminar, we will discuss:

  \begin{enumerate}
    \item The basic Haskell syntax
    \item The underlying aspects of the Haskell type system
    \item Functions and lambdas
    \item Immutability and Laziness
  \end{enumerate}
  }
\end{frame}

\end{document}

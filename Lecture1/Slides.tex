\documentclass[10pt,pdf,utf8,russian,aspectratio=169]{beamer}
\usepackage[T2A]{fontenc}
\usetheme{Copenhagen}
\usepackage{setspace}
\usepackage{amsmath}
\usepackage{pgfplots}
\usepackage[utf8]{inputenc}
\usepackage{tikz-cd}
\usepackage[all, 2cell]{xy}
\usepackage{amssymb}
\usepackage{verba tim}
\usepackage[all]{xy}
\usepackage{tikz}
\usepackage{bussproofs}
\usepackage{dsfont}
\usepackage{mathabx}
\usepackage{animate}
\usetikzlibrary{graphs}
\usetikzlibrary{arrows}
\usepackage{hyperref}
\usepackage[english,russian]{babel}
\usepackage{listings}
\usepackage{color}
\usepackage{tikz}
\usepackage{listings}
\newtheorem{defin}{Definition}
\newtheorem{theor}{Theorem}
\newtheorem{prop}{Proposition}
\title{Functional programming, Seminar No. 1}
\author{Danya Rogozin \\ Lomonosov Moscow State University, \\ Serokell O\"{U}}
\date{Higher School of Economics \\ Faculty of Computer Science}
\begin{document}
\maketitle

\begin{frame}
  \frametitle{General words on Haskell}

  \begin{itemize}
    \item The language is named after American logician Haskell Curry
    \item First implementation: 1990
    \item The language standard: Haskell2010
    \item Default compiler: Glasgow Haskell compiler
    \item Haskell is a strongly-typed, polymorphic, and purely functional programming language
  \end{itemize}
\end{frame}

\begin{frame}
  \frametitle{The Haskell Platform installation}

  There are several ways to install the Haskell platform on Mac:

  \begin{itemize}
    \item Download the \verb".pkg" file and install the corresponding package
    \item Run the script \verb"curl -sSL https://get.haskellstack.org/ | sh"
    \item Install ghc, stack, and cabal via Homebrew
  \end{itemize}

  Choose any way you prefer. All those ways are equivalence to each other.

  \vspace{\baselineskip}

 I'm a Mac user, but I believe that you'll manage to install the Haskell Platform on NixOs/Windows/Linux/etc quite quickly.
\end{frame}

\begin{frame}
  \frametitle{GHC}

  \begin{itemize}
    \item GHC is a default Haskell compiler as we told above
    \item GHC is an open-source project. Don't hesistate to contribute!
    \item GHC is mostly implemented on Haskell
    \item GHC development is produced under the GHC Steering committee control
    \item Very roughly, compiling pipeline is arranged as follows: \\ parsing $\Rightarrow$ compile-time (type-checking mostly)
    $\Rightarrow$ runtime (program execution)
  \end{itemize}
\end{frame}

\begin{frame}
  \frametitle{GHCi}

  \begin{itemize}
    \item GHCi is a Haskell interpreter based on GHC
    \item One may run GHCi with a quite simple command \verb"ghci" on a shell
    \item You play with GHCi as a calculator, the ordinary arithmetical operators are written in a usual way
    \item Take a look at the GHCi chapter in the GHC User's Guide to be familiar with GHCi closely
  \end{itemize}
\end{frame}

\begin{frame}
  \frametitle{Cabal}

  \begin{itemize}
    \item Cabal is a system of library and dependency management
    \item A \verb".cabal" file describe the version of a package and its dependencies
    \item Cabal is also a packaging tool
    \item Cabal is known as a reason of so-called dependency hell
  \end{itemize}
\end{frame}

\begin{frame}
  \frametitle{Stack}

  \begin{itemize}
    \item Stack is a cross-platform build tool for Haskell projects
    \item Stack allows one to
    \begin{itemize}
      \item install packages and version of GHC (and their concrete versions) you need
      \item build, execute, and test projects
      \item reproduce builds
      \item create an isolated location
    \end{itemize}
  \end{itemize}
\end{frame}

\begin{frame}
  \frametitle{Snapshots}

  \begin{itemize}
    \item Snapshot is a curated package set used by Stack
    \item Stackage is a stable repository that stores snapshots
    \item Resolver is a reference to a required snapshot
    \item Let us take a look at the screenshot from Stackage:

    TODO: Snapshots.jpg
  \end{itemize}
\end{frame}

\begin{frame}
  \frametitle{Ecosystem encapsulation}

  The Haskell ecosystem encapsulation might be described as the following sequence:

  TODO: visualise this story somehow
\end{frame}

\begin{frame}
  \frametitle{Creating a Haskell project via Stack}
  \begin{itemize}
    \item Figure out how to call your project and run the script \verb"stack new <projectname>"
    \item You will see the following story after the command \verb"tree ." in the project directory:

    TODO: Tree.jpg
  \end{itemize}
\end{frame}

\begin{frame}
  \frametitle{\verb"stack.yaml"}

  Let us discuss dependencies files in a Haskell project. First of all, we observe the \verb"stack.yaml" file:

  TODO: StackYaml.jpg
\end{frame}

\begin{frame}
  \frametitle{Cabal file}

  As we told above, the \verb".cabal" file describe the relevant version of a project and its dependencies:

  TODO: Cabal.jpg
\end{frame}

\begin{frame}
  \frametitle{\verb"package.yaml"}

  The \verb"package.yaml" generates automatically from the \verb"stack.yaml" and \verb".cabal" files:

  TODO: PackageYaml.jpg
\end{frame}

\begin{frame}
  \frametitle{Building and running a project}

The following commands are crucially important:

  \begin{itemize}
    \item \verb"stack build"
    \item \verb"stack run"
    \item \verb"stack exec"
    \item \verb"stack ghci"
    \item \verb"stack clean"
  \end{itemize}

  The roles of these commands follow their names which are quite self-explanatory.
\end{frame}

\begin{frame}
  \frametitle{Hackage}

  According to its description, 'Hackage is the Haskell community's central package archive of open source software`.

  \begin{itemize}
    \item Webpage: \verb"https://hackage.haskell.org"
    \item Browsing packages
    \item Simplified package search
    \item Current uploads
  \end{itemize}

  TODO: StackScreen.jpg
\end{frame}

\begin{frame}
  \frametitle{Hoogle}

  Hoogle is a sort of Haskell search engine. Webpage: \verb"https://hoogle.haskell.org".

  TODO: Hoogle.jpg
\end{frame}

\begin{frame}
  \frametitle{Summary}

  We observed today such topics as

  \begin{enumerate}
    \item General aspects of GHC and GHCi
    \item The Haskell Platform installation
    \item Dependency management via Stack and Cabal
  \end{enumerate}

  \vspace{\baselineskip}

  On the next seminar, we will discuss:

  \begin{enumerate}
    \item The basic Haskell syntax
    \item The underlying aspects of the Haskell type system
    \item Functions and lambdas
    \item Immutability and Laziness
  \end{enumerate}
\end{frame}

\end{document}

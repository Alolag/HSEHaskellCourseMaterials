\documentclass[10pt]{beamer}
\usepackage[cache=false]{minted}
\usepackage[utf8x]{inputenc}
\usepackage{hyperref}
\usepackage{fontawesome}
\usepackage{graphicx}
\usepackage[english,ngerman]{babel}
\usepackage{bussproofs}

% ------------------------------------------------------------------------------
% Use the beautiful metropolis beamer template
% ------------------------------------------------------------------------------
\usepackage[T1]{fontenc}
\usepackage{fontawesome}
\usepackage{FiraSans}
\newtheorem{defin}{Definition}
\newtheorem{theor}{Theorem}
\newtheorem{prop}{Proposition}
\mode<presentation>
{
  \usetheme[progressbar=foot,numbering=fraction,background=light]{metropolis}
  \usecolortheme{default} % or try albatross, beaver, crane, ...
  \usefonttheme{default}  % or try serif, structurebold, ...
  \setbeamertemplate{navigation symbols}{}
  \setbeamertemplate{caption}[numbered]
  %\setbeamertemplate{frame footer}{My custom footer}
}

% ------------------------------------------------------------------------------
% beamer doesn't have texttt defined, but I usually want it anyway
% ------------------------------------------------------------------------------
\let\textttorig\texttt
\renewcommand<>{\texttt}[1]{%
  \only#2{\textttorig{#1}}%
}

% ------------------------------------------------------------------------------
% minted
% ------------------------------------------------------------------------------


% ------------------------------------------------------------------------------
% tcolorbox / tcblisting
% ------------------------------------------------------------------------------
\usepackage{xcolor}
\definecolor{codecolor}{HTML}{FFC300}

\usepackage{tcolorbox}
\tcbuselibrary{most,listingsutf8,minted}

\tcbset{tcbox width=auto,left=1mm,top=1mm,bottom=1mm,
right=1mm,boxsep=1mm,middle=1pt}

\newtcblisting{myr}[1]{colback=codecolor!5,colframe=codecolor!80!black,listing only,
minted options={numbers=left, style=tcblatex,fontsize=\tiny,breaklines,autogobble,linenos,numbersep=3mm},
left=5mm,enhanced,
title=#1, fonttitle=\bfseries,
listing engine=minted,minted language=r}


% ------------------------------------------------------------------------------
% Listings
% ------------------------------------------------------------------------------
\definecolor{mygreen}{HTML}{37980D}
\definecolor{myblue}{HTML}{0D089F}
\definecolor{myred}{HTML}{98290D}

\usepackage{listings}

% the following is optional to configure custom highlighting
\lstdefinelanguage{XML}
{
  morestring=[b]",
  morecomment=[s]{<!--}{-->},
  morestring=[s]{>}{<},
  morekeywords={ref,xmlns,version,type,canonicalRef,metr,real,target}% list your attributes here
}

\lstdefinestyle{myxml}{
language=XML,
showspaces=false,
showtabs=false,
basicstyle=\ttfamily,
columns=fullflexible,
breaklines=true,
showstringspaces=false,
breakatwhitespace=true,
escapeinside={(*@}{@*)},
basicstyle=\color{mygreen}\ttfamily,%\footnotesize,
stringstyle=\color{myred},
commentstyle=\color{myblue}\upshape,
keywordstyle=\color{myblue}\bfseries,
}


% ------------------------------------------------------------------------------
% The Document
% ------------------------------------------------------------------------------
\title{Functional programming, Seminar No. 5}
\author{Danya Rogozin \\ Lomonosov Moscow State University, \\ Serokell O\"{U}}

\vspace{\baselineskip}

\date{Higher School of Economics \\ The Department of Computer Science}

\begin{document}

\maketitle

\begin{frame}
\frametitle{Intro}

On the previous seminar, we
\begin{itemize}
\item introduced data types, new types, records, and type synonyms
\item told about right and left folds
\item discussed lazy evaluation enforcing
\end{itemize}

\onslide<2->{
    Today we
    \begin{itemize}
    \item motivate and introduce functors
    \item generalise left and right folds with the type class \verb"Foldable"
    \end{itemize}
}
\end{frame}

\section{Functor}

\begin{frame}[fragile]
\frametitle{Motivation}

\begin{itemize}
\item Let us take a look at these functions
    \begin{minted}{haskell}
    map :: (a -> b) -> [a] -> [b]
    map _ [] = []
    map f (x : xs) = f x : map f xs

    mapMaybe :: (a -> b) -> Maybe a -> Maybe b
    mapMaybe _ Nothing = Nothing
    mapMaybe f (Just x) = Just (f x)
    \end{minted}
\item One has the same pattern, a unary function carries through a computational context 
(lists and optional values)
\item This idea has a generalisation with the type class \verb"Functor", 
a Haskell counterpart of categorical functor 
\end{itemize}
\end{frame}

\begin{frame}[fragile]
\frametitle{Here comes the Functor}

Instances of the type class \verb"Functor" are type constructors that has kind \verb"* -> *":

\begin{minted}{haskell}
class Functor (f :: * -> *) where
  fmap :: (a -> b) -> f a -> f b

instance Functor (Maybe a) where
  fmap _ Nothing = Nothing
  fmap f (Just x) = Just (f x)

instance Functor [a] where
  map _ [] = []
  map f (x : xs) = f x : map f xs
\end{minted}
\end{frame}

\begin{frame}[fragile]
\frametitle{The full definition of a functor}

\begin{minted}{haskell}
class Functor (f :: * -> *) where
  fmap        :: (a -> b) -> f a -> f b
  (<$)        :: a -> f b -> f a
  (<$)        =  fmap . const

infixl 4 <$>, <$ 

(<$>) :: Functor f => (a -> b) -> f a -> f b
(<$>) = fmap

void :: Functor f => f a -> f ()
void x = () <$ x
\end{minted}
\end{frame}

\begin{frame}[fragile]
\frametitle{The \verb"Functor" instances for two-parametric types}

Let us take a look at the \verb"Functor" for type constructors that have 
kind \verb"* -> * -> *"

\begin{minted}{haskell}
instance Functor ((,) a) where
  fmap f (x,y) = (x, f y)

instance Functor ((->) r) where
  fmap = (.)

instance Functor (Either a) where
  fmap _ (Left x) = Left x
  fmap f (Right y) = Right (f y)
\end{minted}
\end{frame}

\begin{frame}[fragile]
\frametitle{The \verb"Functor" laws}

Functor has the following axioms:

\begin{minted}{haskell}

fmap id fx = fx

fmap (f . g) fx = (fmap f . fmap g) fx
\end{minted}
\end{frame}

\begin{frame}[fragile]
\frametitle{The \verb"Functor" laws. Example}
Let us check that the list data type is a Functor. In other words, let us check that
the \verb"Functor" instance satisfies required conditions.

\begin{minted}{haskell}

fmap id [] = map id [] = []
fmap id (x : xs) =
  id x : fmap id xs =
  x : fmap id xs =      -- Induction hypothesis
  x : xs

fmap (f . g) [] = []
fmap (f . g) (x : xs) =
  (f . g) x : fmap (f . g) xs =   -- Induction hypothesis
  (f . g) x : (fmap f . fmap g) xs = 
  f (g x) : fmap f (fmap g xs)
\end{minted}
\end{frame}

\section{Monoid}

\begin{frame}[fragile]
\frametitle{The \verb"Monoid" definition}

\begin{minted}{haskell}
class Semigroup a where
  (<>) :: a -> a -> a

class Semigroup a => Monoid a where
  mempty :: a
  mappend :: a -> a -> a
  mappend = (<>)
  mconcat :: [a] -> a
  mconcat = foldr mappend mempty
\end{minted}

\vspace{\baselineskip}

The operation in a semigroup should associative and \verb"mempty" is a neutral element
\begin{minted}{haskell}
a <> (b <> c) = (a <> b) <> c
a <> mempty = a = mempty <> a
\end{minted}
\end{frame}

\begin{frame}[fragile]
\frametitle{The \verb"Monoid" instances}
\begin{minted}{haskell}
instance Semigroup [a] where
  (<>) = (++)

instance Monoid [a] where
  mempty = []
\end{minted}

It is clear that \verb"(++)" is an associative operation and \verb"[]" is neutral.

\end{frame}

\begin{frame}[fragile]
\frametitle{Numbers and Booleans as monoids}

\begin{minted}{haskell}
newtype Sum a = Sum { getSum :: a }
  deriving (Show, Eq, Show)

newtype All = All { getAll :: Bool }
  deriving (Show, Eq, Show)

instance Num a => Semigroup (Sum a) where
  Sum a <> Sum b = Sum (a + b)

instance Semigroup All where
  All a <> All b = All (a && b)

instance Num a => Monoid (Sum a) where
  mempty = Sum 0

instance Monoid All where
  mempty = All True
\end{minted}
\end{frame}

\section{Foldable}

\begin{frame}[fragile]
\frametitle{The \verb"Foldable" type class}
\begin{minted}{haskell}
class Foldable t where
  {-# MINIMAL foldMap | foldr #-}
  fold :: Monoid m => t m -> m
  fold = foldMap id

  foldMap :: Monoid m => (a -> m) -> t a -> m
  foldMap f = foldr (mappend . f) mempty

  foldr :: (a -> b -> b) -> b -> t a -> b
  foldr f z t = appEndo (foldMap (Endo . f) t) z

  foldl :: (b -> a -> b) -> b -> t a -> b
    foldl f z t = 
      appEndo (getDual (foldMap (Dual . Endo . flip f) t)) z
\end{minted}
\end{frame}

\begin{frame}[fragile]
\frametitle{Useful functions for foldable data types}
Here we provide type signatures only:
\begin{minted}{haskell}
toList :: Foldable t => t a -> [a]

null :: Foldable => t a -> Bool

length :: Foldable t => t a -> Int

elem :: (Eq a, Foldable t) => a -> t a -> Bool

maximum :: (Ord a, Foldable t) => t a -> a

sum, product :: (Num a, Foldable t) => t a -> a
\end{minted}
\end{frame}


\section{Applicative functors}

\begin{frame}[fragile]
\frametitle{Motivation}

\begin{itemize}
\item It is clear that we would like to have something like \verb"fmap" for functions that
have an arbitrary arity:

\begin{minted}{haskell}
fmap2 
  :: (a -> b -> c) 
  -> f a -> f b -> f c
fmap3 
  :: (a -> b -> c -> d) 
  -> f a -> f b -> f c -> f d
fmap4 
  :: (a -> b -> c -> d -> e) 
  -> f a -> f b -> f c -> f d -> e
...
\end{minted}
\item That is, one needs to generalise a functor
\item The solution is the \verb"Applicative" type class that extends \verb"Functor"
\end{itemize}
\end{frame}

\begin{frame}[fragile]
\frametitle{The \verb"Applicative" class}

\begin{minted}{haskell}
class Functor f => Applicative f where
    {-# MINIMAL pure, ((<*>) | liftA2) #-}
    pure :: a -> f a

    (<*>) :: f (a -> b) -> f a -> f b
    (<*>) = liftA2 id  -- the same as liftA2 ($)

    liftA2 :: (a -> b -> c) -> f a -> f b -> f c
    liftA2 f x = (<*>) (fmap f x)
\end{minted}
\end{frame}

\begin{frame}[fragile]
\frametitle{The \verb"Applicative" class. Example}
\begin{minted}{haskell}
instance Applicative (Maybe a) where
  pure = Just
  Nothing <*> _ = Nothing
  _ <*> Nothing = Nothing
  Just f <*> Just x = Just (f x)

instance Applicative [a] where
  pure x = [x]
  fs <*> fx = [ f x | f <- fs, x <- xs] 
\end{minted}
\end{frame}

\begin{frame}[fragile]
\frametitle{The \verb"Applicative" laws}
\begin{minted}{haskell}
fmap f x = pure f <*> x

pure id <*> v = v  -- identity

pure (.) <*> u <*> v <*> w = u <*> (v <*> w) -- haskell

pure f <*> pure x = pure (f x) -- homomorphism

u <*> pure y = pure ($ y) <*> u -- interchange
\end{minted}
Let us check some of these laws for the \verb"Maybe" data type on a whiteboard
\end{frame}

\begin{frame}[fragile]
\frametitle{The \verb"Applicative" class. The list problem}
\begin{itemize}
\item The list data type might have an alternative \verb"Applicative" instance
\item One has the function called \verb"zipWith":
\begin{minted}{haskell}
zipWith :: (a -> b -> c) -> [a] -> [b] -> [c]
zipWith _ [] _ = []
zipWith _ _ [] = []
zipWith f (x:xs) (y:ys) = f x y : zipWith f xs ys
\end{minted}
\item The signature of \verb"zipWith" corresponds to the signature of \verb"liftA2"
\item On the other hand, we cannot have two instances for the same data type
\end{itemize}
\end{frame}

\begin{frame}[fragile]
\frametitle{The \verb"Applicative" class. The list problem}

Let us introduce the following data type:
\begin{minted}{haskell}
newtype ZipList a = ZipList { getZipList :: [a]}
\end{minted}

\vspace{\baselineskip}

The instance is implemented via \verb"zipWith":
\begin{minted}{haskell}
instance Applicative (ZipList a) where
  liftA2 f (ZipList xs) (ZipList ys) = 
    ZipList (zipWith f xs ys)
  zipF <*> zipX = liftA2 ($)
  pure = ???
\end{minted}

\vspace{\baselineskip}

How to implement \verb"pure" to preverse the identity law? 

\end{frame}

\section{Traversable}

\begin{frame}[fragile]
\frametitle{The \verb"Traversable" definition}

\begin{minted}{haskell}
class (Functor t, Foldable t) => Traversable t where
  traverse :: Applicative f => (a -> f b) -> t a -> f (t b)
  sequenceA :: Applicative f => t (f a) -> f (t a)
  {-# MINIMAL traverse | sequenceA #-}
\end{minted}
\end{frame}

\section{Summary}

\end{document}